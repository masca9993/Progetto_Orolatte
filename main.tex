\documentclass[10 pt]{extarticle}

\usepackage{amsmath}
\usepackage{amsfonts}
\usepackage{mathtools}
\usepackage{xparse}
\usepackage{physics}
\usepackage{geometry}
\usepackage{array}
\usepackage{siunitx}
\usepackage{titling}
\usepackage{graphicx}
\usepackage{tocloft}
\usepackage{color}
\usepackage{colortbl}
\usepackage{xcolor}
\usepackage{float}
\usepackage{enumerate}
\usepackage{tabu}
\usepackage{changepage}
\usepackage{lipsum}
\usepackage{mathptmx}
\usepackage{makecell}
\usepackage{adjustbox}
\usepackage[11pt]{moresize}
\usepackage[utf8]{inputenc}
\usepackage{listings}
\graphicspath{ {images/} }
\renewcommand{\cftpartleader}{\cftdotfill{\cftdotsep}}
\renewcommand{\cftsecleader}{\cftdotfill{\cftdotsep}}
\renewcommand*\contentsname{Contenuti}
\renewcommand{\abstractname}{Abstract}
\renewcommand{\figurename}{Figura}

\lstset{
frame=tb,
tabsize=4,
showstringspaces=false,
numbers=left,
commentstyle=\color{blue},
keywordstyle=\color{yellow!90!black},
stringstyle=\color{red}
}
\newcolumntype{?}{!{\vrule width 1.5pt}}	%"?" symbolizes bold vertical column in table

\title{Relazione Sito Orolatte}
\author{Biasotto Luca, Faccioli Nicla, Mascari Andrea, Poppi Tommaso}
\date{}

\begin{document}

\maketitle
\begin{abstract}
La seguente relazione descrive come \'e stato realizzato il il sito per la gelateria \emph{Orolatte}, creato nell'ambito del corso di Tecnologie Web, analizzando il codice scritto e discutendo le scelte effettuate.
\end{abstract}
\pagebreak
\tableofcontents
\pagebreak

 \section{Analisi dei Requisiti}
    \subsection{Introduzione}
		Il sito web realizzato come progetto del corso “Tecnologie Web” si occupa di fornire informazioni riguardanti la gelateria Orolatte.
		Come scopo primario il sito permette di esporre i prodotti che vengono venduti all’interno della gelateria, fornendo una descrizione dettagliata per ogni prodotto; inoltre c’è la possibilità, per gli utenti registrati,  di poter ordinare online le torte che vengono esposte nel sito.
		In più viene fornita a disposizione di un amministratore la possibilità di aggiungere i prodotti e la possibilità di modificare le loro specifiche.
    \subsection{Attori}

 \pagebreak
 \section{Progettazione}
    \subsection{Struttura del Sito}
		Le informazioni fondamentali del sito compaiono sempre nella prima metà della pagina, in modo che l'utente sia sempre in grado di rispondere alle tre domande principali:
		\begin{itemize}
			\item Dove sono? Il logo è sempre ben visibile nell'header e i titoli di ciascuna pagina sono scritti dal particolare al generale, in modo che venga sempre mantenuto il contenuto informativo essenziale. Inoltre in ogni pagina sono presenti le breadcrumbs che mostrano all'utente il percorso seguito per arrivare alla pagina attuale.
			\item Di cosa si tratta? L'header non occupa spazio eccessivo proprio per permettere a parte del contenuto della pagina di comparire "above the fold" e dare quindi un'indicazione di cosa la pagina stessa contenga.
			\item Dove posso andare? Anche in questo caso la barra di navigazione è contenuta nell'header in modo che sia subito visibile quali sono le pagine che possono essere raggiunte
		\end{itemize}
    \subsection{Home}
		La pagina \emph{Home} \'e la prima ad essere vista dagli utenti e ha lo scopo di offrire una presentazione chiara e sintetica dell'attivit\'a commerciale rappresentata dal sito.
		Contiene dei paragrafi di testo il cui contenuto \'e una breve presentazione della gelateria e dei suoi prodotti, e due immagini (di dimensioni molto contenute) per rendere la presentazione pi\'u accattivante.
    \subsection{Prodotti}
		La pagina prodotti è strutturata in modo da soddisfare le ricerche di tutti i tipi di utenti utilizzando tre filtri di ricerca:
		il filtro \emph{tutti} è paragonato alla metafora della pesca con la rete, cioè è per gli utenti che vogliono esaminare un pò tutto all’interno dell’argomento;
		mentre i filtri \emph{torte} e \emph{gelati}  sono stati creati per gli utenti che hanno un’idea su cosa stanno cercando e si aspettano di trovar qualcosa di nuovo durante il processo esplorativo (trappola per aragoste).
		Ogni prodotto viene rappresentato all’interno di un box dove viene mostrata un’immagine del prodotto con descrizione e inoltre per le torte viene aggiunto il prezzo e la possibilità di aggiungere la torta al carrello:
		Per dare sempre un’idea chiara all’utente di cosa c’è all’interno del suo carrello, senza che l’utente si sposti continuamente dalla pagina prodotti alla pagina carrello, abbiamo pensato di aggiungere alla pagina prodotti una sezione che riassume ciò che è presente all’interno del carrello, così da non scomodare l’utente durante il suo processo esplorativo, e riducendo così il numero di click da effettuare.
		Quando si effettua il login con l'account admin, la pagina prodotti si trasforma e dà la possibilità all’amministratore di poter aggiungere nuovi prodotti e poter rimuovere o modificare i prodotti già presenti all’interno del sito.
    \subsection{Contatti}
		La pagina contatti è utile per gli utenti che vogliono ritrovare un elemento informativo utile (boa di segnalazione).

 \pagebreak
 \section{Realizzazione}
    \subsection{Struttura}
    \subsection{Presentazione}
    \subsection{Comportamento}
 \pagebreak
 \section{Testing}
	Per verificare il corretto funzionamento delle pagine del sito ed il loro livello di accessibilità, si è fatto uso di diversi strumenti software disponibili in rete.
	Il primo di questi è stato WAVE (Web Accessibility Evaluation Tool) \footnote{https://wave.webaim.org/}, il quale ci ha consentito di individuare eventuali errori di accessibilità all'interno delle singole pagine.

	In fase di sviluppo, a tutti i file php sono state aggiunte le seguenti righe di codice:

	\begin{lstlisting}[language=php, caption={php code using listings}]
		ini_set("display_errors", 1);
		ini_set("display_startup_errors", 1);
		error_reporting(E_ALL); 
	\end{lstlisting}

	Questo ha consentito di stampare a schermo eventuali errori o warning, permettendo quindi di correggerli.
	Sono state rimosse dalla versione finale in modo da non appesantire inutilmente il sito, in quanto non veniva più riportato alcun tipo di avvertimento.

%\begin{thebibliography}
%\end{thebibliography}
\end{document}
